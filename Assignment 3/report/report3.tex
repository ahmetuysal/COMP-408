%%%%%%%%%%%%%%%%%%%%%%%%%%%%%%%%%%%%%%%%%
% University/School Laboratory Report
% LaTeX Template
% Version 3.1 (25/3/14)
%
% This template has been downloaded from:
% http://www.LaTeXTemplates.com
%
% Original author:
% Linux and Unix Users Group at Virginia Tech Wiki 
% (https://vtluug.org/wiki/Example_LaTeX_chem_lab_report)
%
% License:
% CC BY-NC-SA 3.0 (http://creativecommons.org/licenses/by-nc-sa/3.0/)
%
%%%%%%%%%%%%%%%%%%%%%%%%%%%%%%%%%%%%%%%%%

%----------------------------------------------------------------------------------------
%	PACKAGES AND DOCUMENT CONFIGURATIONS
%----------------------------------------------------------------------------------------

\documentclass{article}

\usepackage[version=3]{mhchem} % Package for chemical equation typesetting
\usepackage{siunitx} % Provides the \SI{}{} and \si{} command for typesetting SI units
\usepackage{graphicx} % Required for the inclusion of images
\usepackage{natbib} % Required to change bibliography style to APA
\usepackage{amsmath} % Required for some math elements 
\usepackage[table,xcdraw]{xcolor}
\usepackage{listings}
\usepackage{hyperref}
\usepackage{xcolor}
\usepackage{color} %red, green, blue, yellow, cyan, magenta, black, white
\definecolor{mygreen}{RGB}{28,172,0} % color values Red, Green, Blue
\definecolor{mylilas}{RGB}{170,55,241}
\lstset{language=Matlab,%
    %basicstyle=\color{red},
    breaklines=true,%
    morekeywords={matlab2tikz},
     backgroundcolor=\color{black!5}, % set backgroundcolor
    keywordstyle=\color{blue},%
    morekeywords=[2]{1}, keywordstyle=[2]{\color{black}},
    identifierstyle=\color{black},%
    stringstyle=\color{mylilas},
    commentstyle=\color{mygreen},%
    showstringspaces=false,%without this there will be a symbol in the places where there is a space
    numbers=left,%
    numberstyle={\tiny \color{black}},% size of the numbers
    numbersep=9pt, % this defines how far the numbers are from the text
    emph=[1]{for,end,break},emphstyle=[1]\color{red}, %some words to emphasise
    %emph=[2]{word1,word2}, emphstyle=[2]{style},    
}

\usepackage[utf8]{inputenc} %Türkçe karakterler
\usepackage{floatrow}
\usepackage{geometry}
\usepackage{caption}
\usepackage{subcaption}
\geometry{margin=1in}

\renewcommand{\lstlistingname}{Code Snippet}

%\setlength\parindent{0pt} % Removes all indentation from paragraphs

% Table float box with bottom caption, box width adjusted to content
\newfloatcommand{capbtabbox}{table}[][\FBwidth]


%\usepackage{times} % Uncomment to use the Times New Roman font

%----------------------------------------------------------------------------------------
%	DOCUMENT INFORMATION
%----------------------------------------------------------------------------------------

\title{Assignment 3: \\ Face Detection With A Sliding Window\\ COMP 408} % Title

\author{Ahmet \textsc{Uysal}} % Author name

\date{\today} % Date for the report

\begin{document}

\maketitle % Insert the title, author and date

\begin{center}
\begin{tabular}{l r}
Instructor: & Y\"ucel  \textsc{Yemez} % Instructor/supervisor
\end{tabular}
\end{center}

% If you wish to include an abstract, uncomment the lines below
% \begin{abstract}
% Abstract text
% \end{abstract}


\section{Getting Training (HoG) Features}	

To train our Support Vector Machine classifier we need to get HoG samples from many positive and negative face samples. HoG features are calculated with the help of \href{http://www.vlfeat.org/matlab/matlab.html}{VL Feat toolbox}. Positive face images from two different databases are used to get HoG features for face images. HoG features for negative samples are calculated from non-face images of random parts from given non-face image set, in different scales.

\subsection{Positive Face Samples}

\subsubsection{Caltech Face Samples}
 Positive training database of 6,713 cropped 36x36 faces from \href{http://www.vision.caltech.edu/Image_Datasets/Caltech_10K_WebFaces/}{Caltech Web Faces Project} was given with the started code.

\subsubsection{LFW Face Samples}
13,233 face images from  \href{http://vis-www.cs.umass.edu/lfw/}{Labeled Faces in the Wild} database are also used as positive samples. Original provided pictures was 250x250 and colorful. These images are converted to gray-scale and 36x36 with a MATLAB script.

\begin{lstlisting}[caption={MATLAB script for processing images taken from LFW database.},captionpos=b]
data_path = '../data/'; 
lfw_faces_path = fullfile(data_path, 'lfw_faces');
image_files = dir( fullfile( lfw_faces_path, '*.jpg') );
num_images = length(image_files);

for i = 1:num_images
   img = imread(strcat(image_files(i).folder, '\',...
       image_files(i).name));
   if size(img, 3) == 3
        img = rgb2gray(img);
   end
    
   img = imresize(img, [36, 36]);
   imwrite(img, (fullfile(lfw_faces_path, image_files(i).name)))
end
\end{lstlisting}

\subsubsection{Samples Produced by Warping}

\subsubsection{MATLAB Implementation}
\begin{lstlisting}[caption={MATLAB script for processing images taken from LFW database.},captionpos=b]
image_files = dir( fullfile( train_path_pos, '*.jpg'));
num_images = length(image_files);

first_img = imread(strcat(image_files(1).folder, '\', image_files(1).name));
face_images = zeros([size(first_img), num_images], 'like', first_img); 

% store the images to use in warping
for i = 1:num_images
   face_images(:, :, i) = imread(strcat(image_files(i).folder, '\',...
       image_files(i).name)); 
end
\end{lstlisting}

\subsection{Negative Face Samples}
These samples are taken from provided non-face images. Images are randomly chosen from all non-face images and samples are taken from five different scales (0.25, 0.5, 1, 1.5 and 2). Random hog\_template\_size $\times$ hog\_template\_size parts of the scaled images are taken and used in the HoG calculation. Images are turned to gray-scale since all of our positive samples are in gray-scale. In each scaled versions of the randomly selected image $\lfloor \sqrt{w * h / hog\_template\_size^2}\rfloor$ windows are taken from the image where w and h represents dimensions of the image. I added this part to take samples depending on the size of the images since the differ in size.

\subsubsection{MATLAB Implementation}
\begin{lstlisting}[caption={Related part of the get\_training\_features function for getting HoG values of non-face images.},captionpos=b]
image_files = dir( fullfile( train_path_neg, '*.jpg' ));
num_images = length(image_files);
num_samples = 65000;
% counter for stopping at num_samples
sample_count = 0;
% different scales used for extracting, can be changed
scales = [.25, .5, 1, 1.5, 2];
% initialize  negative features matrix with zeros
features_neg = zeros(num_samples, (hog_template_size / hog_cell_size)^2 * 31);
% get num_samples samples    
while sample_count < num_samples
    % randomly select and image to sample windows from given dataset
    rand_img_index = random('unid', num_images);
    rand_img = imread(strcat(image_files(rand_img_index).folder,...
        '\', image_files(rand_img_index).name));
    % convert selected image to grayscale if it's rgb
    if size(rand_img, 3) == 3
        rand_img = rgb2gray(rand_img);
    end
    % take windows for different scales of randomly selected image
    for scale = scales
        % scale the image
        rand_scaled_img = imresize(rand_img, scale);
        % take the dimensions of the image
        [w, h] = size(rand_scaled_img);    
        % if dimensions are smaller than cell size we can't get any sample 
        if w < hog_template_size || h < hog_template_size
            continue
        end
        % determine how many samples will be taken
        num_sample_from_img = floor(sqrt(w * h / hog_template_size^2));
	% take num_sample_from_img  samples from the scaled image
        for i = 1:num_sample_from_img
            % randomly select window location 
            window_x = random('unid', w + 1 - hog_template_size);
            window_y = random('unid', h + 1 - hog_template_size);
            % increase the sample count
            sample_count = sample_count + 1;
            % calculate HoG for selected window
            hog = vl_hog(im2single(rand_scaled_img(...
                window_x:window_x+hog_template_size-1,...
                window_y:window_y+hog_template_size-1)), hog_cell_size);
            % add result to features_neg
            features_neg(i, :) = hog(:);
            % Stop if we react the wanted amount
            if sample_count >= num_samples
                break
            end
        end
        % Stop if we react the wanted amount
        if sample_count >= num_samples
            break
        end
    end
\end{lstlisting}
\section{Training Support Vector Machine using Features}
VL Feat toolbox is also used here to train a support vector machine, using positive and negative HoG features.

\subsection{MATLAB Implementation}

\begin{lstlisting}[caption={My implementation of svm\_training function.},captionpos=b]
function svmClassifier = svm_training(features_pos, features_neg)
% INPUT:
% . features_pos: a N1 by D matrix where N1 is the number of faces and D
%   is the hog feature dimensionality
% . features_neg: a N2 by D matrix where N2 is the number of non-faces and D
%   is the hog feature dimensionality
% OUTPUT:
% svmClassifier: A struct with two fields, 'weights' and 'bias' that are
%       the parameters of a linear classifier

% combine the features in one matrix to give to vl_svmtrain
X = [ features_pos ; features_neg ];
% create the label vector for indicating positive or negative feature 
Y = [ones(1, length(features_pos)) ,  -ones(1, length(features_neg))];
lambda = 0.00001;
% Function wants an D by N matrix (D: feature dimensions, N: feature count)
% so input the transpose of X
[w, b] = vl_svmtrain(X', Y, lambda);
svmClassifier = struct('weights',w,'bias',b);
end
\end{lstlisting}

\section{Testing SVM Classifier on Training Data}


\section{Analyze Classifier Performance on the Test Data using Sliding Windows}

\section{Dimensionality Reduction using PCA}


\end{document}
